\documentclass[a4paper,12pt,abstracton]{scrartcl}
\usepackage[ngerman, english]{babel}
\usepackage[utf8]{inputenc}
\usepackage[T1]{fontenc}
\usepackage{graphicx}
\usepackage{lipsum}
\usepackage{blindtext}
\usepackage{color}
\usepackage{setspace}
\usepackage{hyperref}
\usepackage[printonlyused]{acronym}
\usepackage{amsmath}
\usepackage[export]{adjustbox}
\usepackage{subcaption}
%\usepackage{mwe}

\title{Bachelorarbeit}
\author{Sophia Milanov}
\date{\today}

\begin{document}
\onehalfspacing
\begin{titlepage}
\begin{center}
 
\Large\textbf{Department of Physics and Astronomy\\
University of Heidelberg}

\vspace{15cm}

\normalsize
Bachelor Thesis in Physics\\
submitted by \\
\vspace{0.5cm}
\Large\textbf{Sophia Milanov}\\
\normalsize
\vspace{0.5cm}
born in Düsseldorf (Germany)\\
\vspace{0.5cm}
\Large\textbf{2016}
\normalsize
\newpage
\mbox{}
\thispagestyle{empty}
\newpage
\
\Large\textbf{Hunting for Intermediate-Mass Black Holes in simulated Globular Clusters using Integrals of Motions}

\vspace{18cm}

\normalsize
This Bachelor Thesis has been carried out by Sophia Milanov at the\\
Max Planck Institute for Astronomy in Heidelberg\\
under the supervision of\\
Dr. Glenn van de Ven

\vfill
\end{center}

\end{titlepage}


\begin{abstract}
\blindtext 
\end{abstract}

\newpage

\tableofcontents

\newpage


\section{Introduction}

\subsection{Motivation}
In this thesis we will apply integrals of motion as actions to  \ac{GC}s to investigate possible signatures of \ac{IMBH}s. 
\subsection{What is a globular cluster in the Milky Way?}\label{sec1.2}
\ac{GC}s are self-gravitating, gas-free systems of 10\(^6\) to 10\(^8\) stars which are spherically grouped. There are about 150 of them in the \ac{MW}. As some of the oldest stellar populations in the universe we can obtain much information about the evolution of the \ac{MW}. Formerly seen as very simple system with only one stellar population and without rotation recent research revealed a much higher complexity of these systems. \color{red} nachlesen, wie das genau mit stellar populations und rotation und models ist \color{black}
\\In this \ac{CMD} the visual magnitude is plotted against the B-V color. It's color coded by the mass of the star. A star's position can be interpreted as its evolution stage. Most of the stars are set in the main sequence. They fusion hydrogen in their cores. There are two main sequence lines one upon the other. These occur due to binary systems. These binary systems represent about \color{red} percentage \color{black} \% of the stars in the \ac{GC}. The main sequence turn-off is depending on the age of the system and is used as indicator for such. Beyond this turn off point there are so called blue stragglers which are remnants of stellar collisions (B\&T p.628). Continuing from the turn off point there is the red giant branch consisting of stars still fusing hydrogen but only in a shell surrounding a degenerate helium core. They are inflated with a radius much higher than the main sequence stars but have only a very low temperature. At the end of the red giant branch lies the horizontal branch. Its stars have sun-like masses and burn helium in their core and hydrogen in a surrounding shell. In the lower left corner white dwarfs are located. They are stellar remnants which have burnt all of their resources. \color{red} nochmal über \ac{CMD} durchlesen und alle branches erwähnen \color{black} In 3.1. we will compare the \ac{CMD} to isochrones which describe the actual distribution of stars in the different stages for a given age and metallicity. 
\\In their centre \ac{GC}s could contain an \ac{IMBH}. These could be the missing link between stellar mass black holes and \ac{SMBH} as origin of the \ac{SMBH}. Currently there are two different kinematic methods trying to detect \ac{IMBH}s. As an example there are the unresolved/integrated IFU kinematics which result in a signature of an \ac{IMBH} for NGC 6388 and resolved/discrete kinematics which don't yield \ac{IMBH}s. \color{red} include graph of paolo's talk? \color{black} To get more certain methods we test if we can derive a signature of the \ac{IMBH} by computing and comparing actions of the stars.


%\includegraphics[width=\textwidth]{Plots/color_magnitude_diagram}
\subsection{Actions \& orbits}
Orbits contain all information about the potential of a system in their position and velocity coordinates following Newtons 2nd law. From the orbit distribution function we can draw inferences about the history of the system. Since there are many possible orbits but the stars are only on some of them questions are raised. How do they come on these special orbits? What do these orbits tell us about the evolution of the system? 
\\There are some examples when orbits enabled discoveries or confirmed them: 
\begin{itemize}
\item Seen from the earth Mars' position moves over the sky as a loop. That implies that the earth is not the centre of the universe!
\item Neptune/Uranus \color{red} noch mal nachlesen \color{black}
\item From rotation curves of galaxies we see that stars move faster than expected by mass of luminous matter. There has to be more matter interacting via gravitational forces. This has led to the theory of dark matter.
\item Mercury's orbit differs hugely from calculated Kepler orbit. This is because of its migrating pericentre. Due to the proximity to the sun gravitational forces are so strong that we need to apply general relativity.
\item The \ac{SMBH} Sagittarius A*  was detected by observations of the orbits around the black hole and resulting mass calculations.
\end{itemize}
Some examples for orbit distribution functions are the asteroid belt, the distribution function of moons around planets which allows feedback connections to formation and bonding of different planet types and spiral galaxies where stars of different parts(thin disc, thick disc, bulge, halo) have different orbits (dynamical distinct) and different metallicity (chemical distinct).
\\\\
Actions are integrals of motion and are the distinct description of orbits. They are constant with time. Known for a long time they are extremly difficult to calculate. Actions of our solar system can be calculated easier since we know the potential. With nowadays supercomputers it's finally possible to compute actions of more complex and less explored systems.





\newpage
\section{Method \& Theory}
%\subsection{stellar population in GC}
\subsection{Density and kinematic profiles of globular clusters}\label{sec2.1}
Our investigations of \ac{GCs} in phase space consists in analysing the spatial distribution of stars (density profiles) and the kinematic profiles(such as velocity dispersion and anisotropy profiles). First we test the sphericity of the \ac{GC}. Sphericity implies the usage of analytical methods that are very straight forward, especially for determing the potential of the globular cluster and then the actions in action space.\par \color{red} density profile \color{black} \par The velocity dispersion is the standard deviation of the mean velocity 
\begin{equation}
\sigma_i=\sqrt{\left\langle(v_i-\langle v_i\rangle)^2\right\rangle}=\sqrt{\left\langle v_i^2-\langle v_i\rangle^2\right\rangle} \qquad\qquad i=r,\theta,\phi.
\end{equation} For a spherical system it is best to calculate them in spherical coordinates \(r,\theta,\phi\) respectively \(v_r,v_{\theta},v_{\phi}\). If the \ac{GC} contains an \ac{IMBH} the velocity dispersion towards the centre is increasing. 
\par \color{red} what exactly describes anisotropy? \color{black} To quantify the anisotropy of the system we use the anisotropy parameter \(\beta\) 
\begin{equation}
\beta=1-\frac{\sigma_\theta ^2+\sigma_\phi ^2}{\sigma_r ^2}.
\end{equation} If \(\beta\) is positive the anisotropy is radial, if it is negative the anisotropy is tangential and if \(\beta\approx0\) then the system is isotropic.

\subsection{Orbits}[\label{sec2.2}]
In a dynamical system the mass distribution is described by the form of theoretically existent orbits (\(\vec{x}(t),\vec{v}(t))\). Position and velocity are linked with six coordinates and contain all information about the potential. With Newton's 2nd law we get the connection between potential \(\Phi(\vec{r})\)and acceleration \(\vec{a}\) which is \[\vec{F}(\vec{r})=-\nabla\Phi(\vec{r})=m\cdot\vec{a}.\] Since the system is spherically potential and force only depend on the distance from the centre r. The potential can be derived from the Poisson's equation \begin{equation}
\Delta\Phi(r)=4\pi G \rho(r)
\end{equation}
with the density \(\rho\) depending only on the distance as well. Due to the spherical symmetry the potential can be calculated by 
\begin{equation}
\Phi(r)=-\frac{G}{r}\int_0^r{\mathrm{d}M(r')}-G\int_r^{\infty}{\frac{\mathrm{d}M(r')}{r'}}=-4\pi G\left[\frac{1}{r}\int_0^r\mathrm{d}r'r'^2\rho(r')+\int_r^{\infty}\mathrm{d}r'r'\rho(r')\right]
\end{equation} (Binney \& Tremaine eq. 2.28). We calculate the density by binning the masses on logarithmic equally distributed shells \color{red} write density formula? \color{black} and solve the integrals of the Poisson's equation numerically by using the Gauss-Legendre quadrature \[\int_a^b f(x)dx = \frac{b-a}{2}\sum_{i=1}^n w_i f\left(\frac{b-a}{2}x_i+\frac{a+b}{2}\right)\] where the points \(x_i\) and the weights \(w_i\) are derived from the Legendre polynomials.\\ Since the orbit is described by position and velocity at each time step we use the numerical leapfrog method which is a second-order time reversible integrator. \(X_i\) and \(v_i\) are calculated by 
\begin{align*}
x_{i+1}=x_i+v_i\Delta t+\frac{a_i(x_i)}{2}\Delta t^2 \\
v_{i+1}=v_i+\frac{a(x_{i+1})+a(x_i)}{2}\Delta t.
\end{align*}



\subsection{Actions}
Stars in spherical symmetric potential are fully described by their actions \begin{equation}
J_i=\frac{1}{2\pi}\oint_{\gamma_i}\vec{p}\cdot d\vec{q} \qquad\qquad i=r,\theta,\phi
\end{equation} which are used as coordinates in action space
These actions are integrals of motion. For most potentials actions can't be described analytically. The actions of a spherical system are derived from angular momentum, energy and potential. Only the potential is depending in r. Energy and angular momentum as well as resulting actions are constant over time and orbit. The azimuthal action J\(_\phi\) and the latitudinal action J\(_\theta\) can be evaluated simply. To calculate the radial action J\(_r\) we have to solve an integral numerically. Actions of a spherical potential are found to be \begin{align}
J_\phi=L_z, \\ J_\theta=L-|L_z|, \\ J_r=\frac{1}{\pi} \int_{r_{min}}^{r_{max}} \mathrm{d}r \sqrt{2E-2\Phi(r)-\frac{L^2}{r^2}}.
\end{align}\color{red} Should I write down the derivation of the actions (B\&T p.220) or just the results (B\&T p.221, 3.221,3.223,3.224)?\color{black} \\ The pericenter \(r_{min}\) and the apocenter \(r_{max}\) as well as the guiding star radisu \(r_g\) can be found in the effective potential \begin{equation}
V_L(r)=V(r)+\frac{L^2}{2mr^2}.\  \color{red} \mathrm{m\ should\ be\ left\ out\ right?} \color{black}
\end{equation} \color{red} Bartelman Theo 1 Skript p59 formula 6.27. \\ \color{black} In the peri- and apocenter the effective potential equals the total energy since the stars do not have any kinetic energy there. That results in following function which is to solve: \[\left(\frac{1}{r}\right)^2+\frac{2\cdot (\Phi-E)}{L^2}=0.\] \\ The guiding star radius is the distance at which a star with given total angular momentum would have a circular orbit. This is at the minimum of the effective potential. To get \(r_g\) we have to solve \[r\sqrt{r\frac{\partial\Phi}{\partial r}}-|L|=0\] where \(\sqrt{r\frac{\partial\Phi}{\partial r}}=v_{circ}\) is the circular velocity. This distance is used to have a better comparison of the actions since in the snapshot the stars are at a random position on their orbit.

\color{red} plot v\_eff plot from barthelman skript? \color{black}

\newpage
\section{Analysis}

\subsection{Description of the simulation}
\begin{table}[htbp]
\begin{tabular}{ c | c | c | c | c }
name of the simulation & \# of patches & total mass & mass of the \ac{IMBH} & r\(_\mathrm{eff}\) \\
\hline			
  SIM 1 - IMBH & & & & \\
  SIM 2 - IMBH & & & & \\
  SIM 3 - NOIMBH & & & & \\
  SIM 4 - NOIMBH & & & & \\

\end{tabular}
\caption{Overview of the data of the simulations.}
\end{table}

\color{red}where simulation comes from and what it is \& description of output \color{black}\\
\\
To get a familiar with the simulation we first have a look at the scatter plot of the star positions.
\begin{figure}[htbp] 
  \includegraphics[width=\textwidth,left]{Plots/position_scatter_plot_IMBH1.pdf}
  \caption{Position scatter plot. The stars are distributed spherically with most of the stars in the inner part. It is spread until 100 pc.}
  
  \label{fig:Bild1}
\end{figure}

\color{red} include same plot with velocities!! \color{black}\\
As we said in section \ref{sec2.1} it is important to test the sphericity of a system. We will do this by splitting the \ac{GC} into octants and compare their mass \color{red} or number since we will introduce mass density plots only later \color{black} density \color{red} mass density sphericity plots\color{black}. As you see they're acceptable overlaying \color{red} within their errors. \color{black}
\\
As mentioned in \ref{sec1.2} the \ac{CMD} is showing a star's evolution stage dependend on its position. If you do not know age or metallicity of the system you can fit isochrones on the \ac{CMD}. Isochrones are curves of evolutionary stages of stars having the same age and metallicity but different masses. We plot several to our \ac{CMD} to determine which one fits best. This will give us the age and the metallicity of the system. \color{red} cmd with isocrones plots \color{black}


\subsection{Investigation in phase space}

First we will investigate the \ac{GC} in phase space for the set of simulations that w will use throughout this work. We will start with the velocity dispersion and the anisotropy parameter then we will have a density profile and from that get the potential. \color{red} for all 4 simulations or only for the first with IMBH? or with on plot containing two or all of them? \color{black}

\subsubsection{Velocity dispersion}
With (1) from section \ref{sec2.1} we can calculate the velocity dispersion for each coordinate \(r,\theta,\phi\). For every bin we take the same amount of stars and calculate the dispersion. To compare both simulations we plot the dispersion over the effective radius. The half mass radius of the simulation with \ac{IMBH} is 4.1pc and of the simulation without \ac{IMBH} it is 7.9pc. 
\begin{figure*}[htbp]
	\centering
	\begin{subfigure}{0.475\textwidth}
		\centering
		\includegraphics[width=\textwidth]{Plots/radial_velocity_dispersion.pdf}
		\caption{Radial velocity dispersions}
		\label{[fig:radial_vel_disp]}
	\end{subfigure}
	\hfill
	\begin{subfigure}{0.475\textwidth}
		\centering
		\includegraphics[width=\textwidth]{Plots/azimuthal_velocity_dispersion.pdf}
		\caption{Azimuthal velocity dispersions}
		\label{[fig:azimuthal_vel_disp]}
	\end{subfigure}
	\vskip\baselineskip
	\begin{subfigure}{0.475\textwidth}
		\centering
		\includegraphics[width=\textwidth]{Plots/polar_velocity_dispersion.pdf}
		\caption{Polar velocity dispersions}
		\label{[fig:polar_vel_disp]}
	\end{subfigure}


	\caption{Velocity dispersion profiles as a function of the radius in units of the effective radius \(\mathrm{r_{eff}}\). They are binned in a way that each bin contains the same amount of stars. We can see that the velocity dispersion of the simulation with \ac{IMBH} rises towards the centre whereas the simulation without \ac{IMBH} exhibits a cored profile.}
\end{figure*}
As expected there is a rise in the centre for the simulation with \ac{IMBH}. This is due the high gravitational potential of the \ac{IMBH} which disturbs the dynamics of close stars. 


\subsubsection{Anisotropy}
Anisotropy can be calculated from (2) in \ref{sec2.1}. It is binned the same way as for the velocity dispersion and again dependent on the effective radii.
\begin{figure}
	\centering
	\includegraphics[width=0.475\textwidth]{Plots/anisotropy_parameter_beta.pdf}
	\caption{Anisotropy parameter \(\beta\). Both simulations are radial anisotropic. The simulation with \ac{IMBH} has a peak at 4 effective radii where it is most radial anisotropic. The other simulation is rising until the end. The far more outside a star is the higher its radial anisotropy is.}
	\label{fig:anisotropy_param}
\end{figure}
In the center of both \ac{GCs} there is nearly the same anisotropy. Both are positive and rising. That means the systems are radial anisotropic. The \ac{GC} with \ac{IMBH} is most radial anisotropic in its center at about 4 effective radii. The other \ac{GC} is becoming more radial anisotropic the more far from the centre it is.

\subsubsection{Density profile}
The density profile shows the density of the system over its radius. The bins are chosen so that the radii are equidistant on a logarithmic scale and that they are at least 100 stars per bin to have a reliable stochastic. Outside of the cluster the density is set to 0. In the innermost part the density is set to be as the innermost point.
plots\\
\begin{figure}
\centering
\includegraphics[width=0.475\textwidth]{Plots/density_profile.pdf}
\caption{Mass density profiles. The density in \(\frac{M_{\odot}}{pc^2}\) is plotted against the effective radius. The density of the \ac{GC} with \ac{IMBH} is everywhere larger than the density of the \ac{GC} without \ac{IMBH}. In the centre there is a raise in the density of the \ac{GC} with \ac{IMBH} whereas the other \ac{GC} stays approximately on the same level. Both start decreasing at about 0.5 \(\mathrm{r_eff}\).}
\label{fig:mass_density_profile}
\end{figure}

\subsubsection{Potential}
From the density profile we can compute the potential as described in \ref{sec2.2}. It is composed by the potential given from the stars and if there is one the potential of the \ac{IMBH} expressed as Kepler potential.


\subsection{Investigations of orbits in action space}
wilma class 
\subsubsection{Orbits}
\subsubsection{Actions}
\subsubsection{Integral of motions along orbits}
\newpage
\section{Results \& Discussion}
only triangle plots

\subsection{Actions from different globular clusters}
\subsection{Discussion \& future perspectives}
do the same distinguishing the mass of the stars\\
redo the work with only observational-like data
\newpage
\section{Conclusion}
\newpage 
\section{Acronyms}

\begin{acronym}[SMBHs]
	\acro{CMD}{color magnitude diagram}
	\acro{DFs}{distribution functions}
	\acro{GC}{globular cluster}
	\acro{GCs}{globular clusters}
	\acro{IMBH}{intermediate mass black hole}
	\acro{IMBHs}{intermediate mass black holes}
	\acro{MW}{Milky Way}
	\acro{SMBH}{super massive black hole}
	\acro{SMBHs}{super massive black holes}
\end{acronym}
\end{document}