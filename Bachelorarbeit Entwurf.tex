\documentclass[a4paper,12pt,abstracton]{scrartcl}
\usepackage[ngerman, english]{babel}
\usepackage[utf8]{inputenc}
\usepackage[T1]{fontenc}
\usepackage{graphicx}
\usepackage{lipsum}
\usepackage{blindtext}
\usepackage{color}
\usepackage{setspace}

\title{Bachelorarbeit}
\author{Sophia Milanov}
\date{\today}

\begin{document}
\onehalfspacing
\begin{titlepage}
\begin{center}
 
\Large\textbf{Department of Physics and Astronomy\\
University of Heidelberg}

\vspace{15cm}

\normalsize
Bachelor Thesis in Physics\\
submitted by \\
\vspace{0.5cm}
\Large\textbf{Sophia Milanov}\\
\normalsize
\vspace{0.5cm}
born in Düsseldorf (Germany)\\
\vspace{0.5cm}
\Large\textbf{2016}
\normalsize
\newpage
\mbox{}
\thispagestyle{empty}
\newpage
\
\Large\textbf{Hunting for Intermediate-Mass Black Holes in simulated Globular Clusters using Integrals of Motions}

\vspace{18cm}

\normalsize
This Bachelor Thesis has been carried out by Sophia Milanov at the\\
Max Planck Institute for Astronomy in Heidelberg\\
under the supervision of\\
Dr. Glenn van de Ven

\vfill
\end{center}

\end{titlepage}


\begin{abstract}
\blindtext 
\end{abstract}

\newpage

\tableofcontents

\newpage


\section{Introduction}

\subsection{Motivation}
In this thesis we will apply integrals of motion as actions to  \ac{GC}s to investigate possible signatures of \ac{IMBH}s. 
\subsection{What is a globular cluster in the Milky Way?}\label{sec1.2}
\ac{GC}s are self-gravitating, gas-free systems of 10\(^6\) to 10\(^8\) stars which are spherically grouped. There are about 150 of them in the \ac{MW}. As some of the oldest stellar populations in the universe we can obtain much information about the evolution of the \ac{MW}. Formerly seen as very simple system with only one stellar population and without rotation recent research revealed a much higher complexity of these systems. \color{red} nachlesen, wie das genau mit stellar populations und rotation und models ist \color{black}
\\In this \ac{CMD} the visual magnitude is plotted against the B-V color. It's color coded by the mass of the star. A star's position can be interpreted as its evolution stage. Most of the stars are set in the main sequence. They fusion hydrogen in their cores. There are two main sequence lines one upon the other. These occur due to binary systems. These binary systems represent about \color{red} percentage \color{black} \% of the stars in the \ac{GC}. The main sequence turn-off is depending on the age of the system and is used as indicator for such. Beyond this turn off point there are so called blue stragglers which are remnants of stellar collisions (B\&T p.628). Continuing from the turn off point there is the red giant branch consisting of stars still fusing hydrogen but only in a shell surrounding a degenerate helium core. They are inflated with a radius much higher than the main sequence stars but have only a very low temperature. At the end of the red giant branch lies the horizontal branch. Its stars have sun-like masses and burn helium in their core and hydrogen in a surrounding shell. In the lower left corner white dwarfs are located. They are stellar remnants which have burnt all of their resources. \color{red} nochmal über \ac{CMD} durchlesen und alle branches erwähnen \color{black} In 3.1. we will compare the \ac{CMD} to isochrones which describe the actual distribution of stars in the different stages for a given age and metallicity. 
\\In their centre \ac{GC}s could contain an \ac{IMBH}. These could be the missing link between stellar mass black holes and \ac{SMBH} as origin of the \ac{SMBH}. Currently there are two different kinematic methods trying to detect \ac{IMBH}s. As an example there are the unresolved/integrated IFU kinematics which result in a signature of an \ac{IMBH} for NGC 6388 and resolved/discrete kinematics which don't yield \ac{IMBH}s. \color{red} include graph of paolo's talk? \color{black} To get more certain methods we test if we can derive a signature of the \ac{IMBH} by computing and comparing actions of the stars.


%\includegraphics[width=\textwidth]{Plots/color_magnitude_diagram}
\subsection{Actions \& orbits}
Orbits contain all information about the potential of a system in their position and velocity coordinates following Newtons 2nd law. From the orbit distribution function we can draw inferences about the history of the system. Since there are many possible orbits but the stars are only on some of them questions are raised. How do they come on these special orbits? What do these orbits tell us about the evolution of the system? 
\\There are some examples when orbits enabled discoveries or confirmed them: 
\begin{itemize}
\item Seen from the earth Mars' position moves over the sky as a loop. That implies that the earth is not the centre of the universe!
\item Neptune/Uranus \color{red} noch mal nachlesen \color{black}
\item From rotation curves of galaxies we see that stars move faster than expected by mass of luminous matter. There has to be more matter interacting via gravitational forces. This has led to the theory of dark matter.
\item Mercury's orbit differs hugely from calculated Kepler orbit. This is because of its migrating pericentre. Due to the proximity to the sun gravitational forces are so strong that we need to apply general relativity.
\item The \ac{SMBH} Sagittarius A*  was detected by observations of the orbits around the black hole and resulting mass calculations.
\end{itemize}
Some examples for orbit distribution functions are the asteroid belt, the distribution function of moons around planets which allows feedback connections to formation and bonding of different planet types and spiral galaxies where stars of different parts(thin disc, thick disc, bulge, halo) have different orbits (dynamical distinct) and different metallicity (chemical distinct).
\\\\
Actions are integrals of motion and are the distinct description of orbits. They are constant with time. Known for a long time they are extremly difficult to calculate. Actions of our solar system can be calculated easier since we know the potential. With nowadays supercomputers it's finally possible to compute actions of more complex and less explored systems.





\newpage
\section{Method \& Theory}
%\subsection{stellar population in GC}
\subsection{Observed kinematics in globular clusters}
Our investigations of globular clusters in phase space consists star distribution plots, the velocity dispersion and the anisotropy parameter. First we test the sphericity of the globular cluster. Sphericity implies the usage of analytical methods especially for determing the potential of the globular cluster and the actions in action space. \\ The velocity dispersion is the standard deviation of the mean velocity \[\sigma_i=\sqrt{\left\langle(v_i-\langle v_i\rangle)^2\right\rangle}=\sqrt{\left\langle v_i^2-\langle v_i\rangle^2\right\rangle} \qquad\qquad i=r,\theta,\phi.\] For a spherical system it's best to calculate them in spherical coordinates \(r,\theta,\phi\) respectively \(v_r,v_{\theta},v_{\phi}\). If the globular cluster contains an IMBH the velocity dispersion towards the centre is increasing. To quantify the anisotropy of the system we use the anisotropy parameter \(\beta\) \[\beta=1-\frac{\sigma_\theta ^2+\sigma_\phi ^2}{\sigma_r ^2}.\] If \(\beta\) is positive the anisotropy is radial and if it's negative the anisotropy is tangential.

\subsection{Orbits}
In a dynamical system the mass distribution is described by the form of theoretically existent orbits (\(x(t),v(t))\). Position and velocity are linked with six coordinates and contain all information about the potential. With Newton's 2nd law we get the connection between potential \(\Phi(\vec{r})\)and acceleration \(\vec{a}\) which is \[\vec{F}(\vec{r})=-\nabla\Phi(\vec{r})=m\cdot\vec{a}\]. 

Poisson's equation
density \& potent
\subsection{actions}
\newpage
\section{Analysis}

\subsection{Description of the simulation}
where simulation comes from and what it is \& description of output\\
x-y-z plot how it looks like\\
test of sphericity \& center\\
cmd with isochrones explaining them and showing that they fit with the simulation\\

\subsection{Investigation in phase space}
Paolo class

\subsubsection{Velocity dispersion}
\includegraphics[width=0.4\textwidth]{Plots/radial_velocity_dispersion.pdf}\\
\includegraphics{Plots/azimuthal_velocity_dispersion.pdf}\\
\includegraphics{Plots/polar_velocity_dispersion.pdf}\\
\includegraphics{Plots/anisotropy_parameter_beta.pdf}
aussage\\
plots\\
erklärung physikalisch
\subsubsection{anisotropy}
\subsubsection{Density profile}
plots\\
bestätigung kugelförmig\\
potential daraus
\subsubsection{Potential}

\subsection{Investigations of orbits in action space}
wilma class 
\subsubsection{Orbits}
\subsubsection{Actions}
\subsubsection{Integral of motions along orbits}
\newpage
\section{Results \& Discussion}
only triangle plots

\subsection{Actions from different globular clusters}
\subsection{Discussion \& future perspectives}
do the same distinguishing the mass of the stars\\
redo the work with only observational light data
\newpage
\section{Conclusion}
\end{document}