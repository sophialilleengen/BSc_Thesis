\section{Conclusion}
In this work we investigated simulated \acp{GC} in orbit space, specifically in action space, to figure out if we can use this method to find signatures of \acp{IMBH}. Given four simulations two of which containing an \ac{IMBH} we first examined the properties in color-magnitude and in phase space. Generating the potentials of the simulations from their density profiles we calculated integrals of motion for all the \ac{GC} stars. Due to the spherically symmetric potential of the \acp{GC} only the energy and the total angular momentum as classical integrals of motion and particularly the radial action as best choice of integral of motions were relevant to us. We found a clear signature of \acp{IMBH} in the distribution of stars in the specific radial action versus specific energy plane. More but less clear signatures were found in a distribution of the radial action versus the guiding star radius and in a distribution of the radial action versus the angular momentum. The robustness of these signatures was tested, using a wrong density as a base of our calculations. Further investigations have to be done, i.e. by changing the Kepler potential of the \ac{IMBH} and comparing the simulations with exact identical initial conditions and as only difference the presence of an \ac{IMBH}. Then we will have to think about how to adapt our method for an application to observational data. 